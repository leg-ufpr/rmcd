\documentclass[9pt,a5paper,]{book}
\usepackage{lmodern}
\usepackage{amssymb,amsmath}
\usepackage{ifxetex,ifluatex}
\usepackage{fixltx2e} % provides \textsubscript
\ifnum 0\ifxetex 1\fi\ifluatex 1\fi=0 % if pdftex
  \usepackage[T1]{fontenc}
  \usepackage[utf8]{inputenc}
\else % if luatex or xelatex
  \ifxetex
    \usepackage{mathspec}
  \else
    \usepackage{fontspec}
  \fi
  \defaultfontfeatures{Ligatures=TeX,Scale=MatchLowercase}
\fi
% use upquote if available, for straight quotes in verbatim environments
\IfFileExists{upquote.sty}{\usepackage{upquote}}{}
% use microtype if available
\IfFileExists{microtype.sty}{%
\usepackage{microtype}
\UseMicrotypeSet[protrusion]{basicmath} % disable protrusion for tt fonts
}{}
\usepackage[left=2cm,right=1.2cm,top=1.5cm,bottom=1.5cm]{geometry}
\usepackage{hyperref}
\hypersetup{unicode=true,
            pdftitle={Regression Models for Count Data: beyond the Poisson model},
            pdfauthor={Wagner Hugo Bonat; Walmes Marques Zeviani; Eduardo Elias Ribeiro Jr},
            pdfborder={0 0 0},
            breaklinks=true}
\urlstyle{same}  % don't use monospace font for urls
\usepackage{natbib}
\bibliographystyle{apalike}
\usepackage{longtable,booktabs}
\usepackage{graphicx,grffile}
\makeatletter
\def\maxwidth{\ifdim\Gin@nat@width>\linewidth\linewidth\else\Gin@nat@width\fi}
\def\maxheight{\ifdim\Gin@nat@height>\textheight\textheight\else\Gin@nat@height\fi}
\makeatother
% Scale images if necessary, so that they will not overflow the page
% margins by default, and it is still possible to overwrite the defaults
% using explicit options in \includegraphics[width, height, ...]{}
\setkeys{Gin}{width=\maxwidth,height=\maxheight,keepaspectratio}
\IfFileExists{parskip.sty}{%
\usepackage{parskip}
}{% else
\setlength{\parindent}{0pt}
\setlength{\parskip}{6pt plus 2pt minus 1pt}
}
\setlength{\emergencystretch}{3em}  % prevent overfull lines
\providecommand{\tightlist}{%
  \setlength{\itemsep}{0pt}\setlength{\parskip}{0pt}}
\setcounter{secnumdepth}{5}
% Redefines (sub)paragraphs to behave more like sections
\ifx\paragraph\undefined\else
\let\oldparagraph\paragraph
\renewcommand{\paragraph}[1]{\oldparagraph{#1}\mbox{}}
\fi
\ifx\subparagraph\undefined\else
\let\oldsubparagraph\subparagraph
\renewcommand{\subparagraph}[1]{\oldsubparagraph{#1}\mbox{}}
\fi

%%% Use protect on footnotes to avoid problems with footnotes in titles
\let\rmarkdownfootnote\footnote%
\def\footnote{\protect\rmarkdownfootnote}

%%% Change title format to be more compact
\usepackage{titling}

% Create subtitle command for use in maketitle
\newcommand{\subtitle}[1]{
  \posttitle{
    \begin{center}\large#1\end{center}
    }
}

\setlength{\droptitle}{-2em}
  \title{Regression Models for Count Data: beyond the Poisson model}
  \pretitle{\vspace{\droptitle}\centering\huge}
  \posttitle{\par}
  \author{Wagner Hugo Bonat \\ Walmes Marques Zeviani \\ Eduardo Elias Ribeiro Jr}
  \preauthor{\centering\large\emph}
  \postauthor{\par}
  \date{}
  \predate{}\postdate{}

% Mathematics environments
\usepackage{amssymb}
\usepackage{amsmath}
\usepackage{amstext}

% Customize itemize's enviroments
\usepackage{enumitem}

% Fonts
\usepackage{mathpazo}
\usepackage{eulervm}
\usepackage{inconsolata}
\urlstyle{tt}

% Compact toc
\usepackage{tocloft}

% Footers and headers styles
\usepackage{fancyhdr}
\pagestyle{fancy}
\fancyhf{}
\fancyhead[LE,RO]{\thepage}
\fancyhead[RE]{\scriptsize\leftmark}
\fancyhead[LO]{\scriptsize\rightmark}

% Modify color in Rcodes/chunks (works well only monochrome highlight)
\usepackage{xcolor}
\definecolor{inputcolor}{RGB}{25,25,112}
\usepackage{framed}
\usepackage{fancyvrb}
\DefineVerbatimEnvironment{Highlighting}{Verbatim}{commandchars=\\\{\},
                                                   fontsize=\small}
\definecolor{shadecolor}{RGB}{248,248,248}
\renewenvironment{Shaded}{\color{inputcolor}}{}
\renewcommand{\DataTypeTok}[1]{{#1}}

% Modify style of verbatim (chunks outputs)
\definecolor{outputcolor}{RGB}{139,0,0}
\makeatletter
\def\verbatim@font{\ttfamily \small \color{outputcolor}}%
\makeatother

\begin{document}
\maketitle

\thispagestyle{empty}
\cleardoublepage
\thispagestyle{empty}

\topskip0pt

\begin{flushleft}
  \Large \bf
  Regression Models for Count Data
\end{flushleft}
\vspace*{1.5em}

\begin{flushleft}
Wagner Hugo Bonat\\
\url{www.leg.ufpr.br/~wagner}

Walmes Marques Zeviani\\
\url{www.leg.ufpr.br/~walmes}

Eduardo Elias Ribeiro Jr\\
\url{www.leg.ufpr.br/~eduardojr}
\end{flushleft}
\vspace*{2em}

Laboratório de Estatística e Geoinformação (LEG)\\
\url{http://www.leg.ufpr.br}\\
Departamento de Estatística\\
Universidade Federal do Paraná (UFPR)

Supplementary content: \url{http://www.leg.ufpr.br/rmcd}\\
Contact: \url{rmcd@leg.ufpr.br}
\vspace*{\fill}

\begin{center}
XV EMR - Escola de Modelos de Regressão\\
Goiânia - Goiás, Brasil\\
March 26 to 29, 2017
\end{center}

\clearpage
\thispagestyle{empty}
\pagebreak

\setcounter{page}{1}

{
\setcounter{tocdepth}{1}
\tableofcontents
}
\chapter*{Preface}\label{preface}
\addcontentsline{toc}{chapter}{Preface}

The main goal of this material is to provide a technical support for the
students attending the course ''Regression models for count data: beyond
the Poisson model``, given as part of the XV Brazilian School of
Regression models - March/2017 in Goiânia, Brazil.

The main goal of this course is to present a wider range of statistical
models to deal with count data. In particular, we focus on parametric
and second-moments specified models. We shall present the model
specification along with strategies for model fitting and the associated
\texttt{R} code. Furthermore, this book-course and supplementary
material as \texttt{R} \citep{R2015} code and data sets are available
for the students.

We intend to keep the course in a level suitable for bachelor students
who already attended a course on generalized linear models
\citep{Nelder1972}. However, since the course also covers updated
topics, it can be of interest of postgraduate students and researches in
general.

We designed the course for three hours of tuition. In the first part
(two hours) of the course, we shall present the analysis of count data
based on fully parametric models. After a brief introduction and
motivation on count data, we present the Gamma-Count, Poisson-Tweedie
and COM-Poisson distributions. We explore their properties through a
consideration of dispersion, zero-inflated and heavy tail indexes, and
illustrate their applications with four data analyses. The estimation of
these models based on the likelihood paradigm is discussed along with
\texttt{R} code and worked examples.

In the second part (one hour) of the course, we provide a brief
introduction to the estimating function approach \citetext{\citealp[
]{Jorgensen2004}; \citealp{Bonat2016a}} and discuss models based on
second-moments assumptions in the style of \citet{Wedderburn1974}. In
particular, we focus on the recently proposed Extended Poisson-Tweedie
\citep{Bonat2016b} and the quasi-Poisson models. The estimating function
approach adopted for estimation and inference is presented along with
\texttt{R} code and data examples. The use of the \texttt{R} package
\texttt{mcglm} \citep{Bonat2016c} is discussed for fitting the extended
Poisson-Tweedie model.

We acknowledge our gratitude to the scientific committee of XV Brazilian
regression model school for this opportunity.

\chapter{Introduction}\label{introduction}

The analysis of count data has received attention from the statistical
community in the last four decades. Since the seminal paper published by
Nelder and Wedderburn \citep{Nelder1972}, the class of generalized
linear models (GLMs) have a proeminent role for regression modelling of
normal and non-normal data including count data. The success enjoyed by
the GLM framework comes from its ability to deal with a wide range of
normal and non-normal data. GLMs are fitted by a simple and efficient
Newton score algorithm relying only on second-moment assumptions for
estimation and inference. Furthermore, the theoretical background for
GLMs is well established in the class of dispersion models
\citep{Jorgensen1987, Jorgensen1997} as a generalization of the
exponential family of distributions.

In spite of the flexibility of the GLM class, the Poisson distribution
is the only choice for the analysis of count data in this framework.\\
Thus, in practice there is probably an over-emphasis on the use of the
Poisson distribution for count data. In practice, however, count data
can present other features, namely underdispersion (mean \textgreater{}
variance) and overdispersion (mean \textless{} variance). There are many
different possible causes for departures from the equidispersion.
Furthermore, in practical data analysis a number of these could be
involved. In this short-course we shall discuss some models to deal with
non-equidispersed count data.

One possible cause of under/overdispersion is departure from the Poisson
process. It is well known that the Poisson counts can be interpreted as
the number of events in a given time interval where the arrival's times
are exponential distributed. When this assumption is violated the
resulting counts can be under or overdispersed \citep{Zeviani2014}.
Another possibility and probably more frequent cause of overdispersion
is unobserved heterogeneity of experimental units. It can be due, for
example, to correlation between individual responses, cluster sampling,
ommitted covariates and others.

In general, these departures from the Poisson distribution are
manifested in the raw data as a zero-inflated or heavy-tail count
distributions. Furthermore, it is important to discuss the consequences
of failing to take it into account the under or overdispersion when
analysing count data. In the case of overdispersion, the standard errors
associated with the regression coefficients calculated under the Poisson
assumption are too optimistic and associated hypothesis tests will tend
to give false positive results by incorrectly rejecting null hypotheses.
The opposite situation will appear in case of underdispersed data. In
both cases, the Poisson model provides unreliable standard errors for
the regression coefficients and hence potentially misleading inferences.
However, the regression coefficients are still consistently estimated.

The strategies for constructing alternative count distributions are
related with the causes of the non-equidispersion. When departures from
the Poisson process are plausible the class of duration time models
\citep{Winkelmann2003} can be employed. This class of models basically
changes the distribution of the time between events from the exponential
to more general distributions, like gamma and inverse Gaussian. In this
course, we shall discuss one example of this approach, namely, the
Gamma-Count distribution \citep{Zeviani2014}. This distribution assumes
that the time between events is gamma distributed, thus it can deal with
under and overdispersed count data.

On the other hand, if unobserved heterogeneity is present its in general
implies extra variability and consequently overdispersed count data. In
this case, a Poisson mixtures is commonly applied. This approach
consists of include random effects on the observation level, and thus
take into account the unobserved heterogeneity. Probably, the most
popular example of this approach is the negative binomial model, that
corresponds to a Poisson-gamma mixtures. In this course, we shall
present the Poisson-Tweedie family of distributions, which in turn
corresponds to Poisson-Tweedie mixtures
\citep{Bonat2016b, Jorgensen2014}. Finally, a third approach to deal
with non-equidispersed count data consists of generalize the Poisson
distribution by adding an extra parameter to model under and
overdispersion. Such a generalization can be done using the class of
weighted Poisson distributions \citep{DelCastillo1998}. One popular
example of this approach is the Conway--Maxwell--Poisson distribution
(COM-Poisson) \citep{Sellers2010}. The COM-Poisson is a member of the
exponential family, has the Poisson and geometric distributions as
special cases and the Bernoulli distribution as a limiting case. It can
deal with both under and overdispersed count data. Thus, given the nice
properties of the COM-Poisson distribution for handling count data, we
choose to present this model as part of this short course.

In this short course we shall highlight and compare the ability of these
distributions to deal with count data through a consideration of
dispersion, zero-inflated and heavy tail indexes. Furthermore, we
specify regression models and illustrate their application with four
worked examples.

In Chapter \(2\) after introduce the dispersion, zero-inflated and heavy
tail indexes we present the properties and regression models based on
the Poisson, Gamma-count, Poisson-Tweedie and COM-Poisson distributions.
Estimation and inference for these models based on the likelihood
paradigm are described in Chapter \(3\). In Chapter \(4\), we extend the
Poisson and Poisson-Tweedie by specifying models based only on
second-moments assumptions and present the related estimation and
inference based on the estimating function approach. Chapter \(5\)
presents four worked examples. Finally, in Chapter \(6\) we discuss the
general methods and propose some topics for future works.

\chapter{Count distributions: properties and regression
models}\label{count-distributions-properties-and-regression-models}

In this chapter, we present the probability mass function and discuss
the main properties of the Poisson, Gamma-Count, Poisson-Tweedie and
COM-Poisson distributions. Let \(Y\) be a count random variable and
\(\mathrm{E}(Y)\) and \(\mathrm{var}(Y)\) denote its mean and variance,
respectively. In order to explore and compare the flexibility of the
models in discussion, we introduce the
dispersion\textasciitilde{}(\(\mathrm{DI}\)),
zero-inflation\textasciitilde{}(\(\mathrm{ZI}\)) and
heavy-tail\textasciitilde{}(\(\mathrm{HT}\)) indexes, which are
respectively given by

\begin{equation}
\mathrm{DI} = \mathrm{Var}(Y)/\mathrm{E}(Y), \quad \mathrm{ZI} = 1 + \frac{\log \mathrm{P}(Y = 0)}{\mathrm{E}(Y)} \quad \text{and} \quad
\mathrm{HT} = \frac{\mathrm{P}(Y=y+1)}{\mathrm{P}(Y=y)}\quad \text{for} \quad y \to \infty. 
\end{equation}

These indexes are defined in relation to the Poisson distribution. Thus,
the dispersion index indicates underdispersion for \(\mathrm{DI} < 1\),
equidispersion for \(\mathrm{DI} = 1\) and overdispersion for
\(\mathrm{DI} > 1\). Similarly, the zero-inflated index is easily
interpreted, since \(ZI < 0\) indicates zero-deflation, \(ZI = 0\)
corresponds to no excess of zeroes and \(\mathrm{ZI} > 0\) indicates
zero-inflation. Finally, \(\mathrm{HT} \to 1\) when \(y \to \infty\)
indicates a heavy tail distribution. In what follows, we shall present
the Poisson, Gamma-Count, Poisson-Tweedie and COM-Poisson distributions
and explore their properties to deal with non-equidispersed count data.

\section{Poisson distribution}\label{poisson-distribution}

The Poisson distribution is a notorious discrete discrete distribution
by its very interesting properties. First, it has a dual interpretation
as a natural exponential family and as an exponential dispersion model.
The Poisson distribution denoted by \(P(\lambda)\) with mean \(\lambda\)
has probability function

\begin{eqnarray}
p(y;\lambda) &=& \frac{\lambda^y}{y!}\exp{-\lambda}
         &=& \frac{1}{y!} \exp{\phi y -  \exp{\phi} }, \quad y \in \mathbb{N}_{0},  
\end{eqnarray}

where \(\phi = \log{\lambda} \in \mathbb{R}\). Hence the Poisson is a
natural exponential family with cumulant generator
\(\kappa(\phi) = \exp{\phi}\). Thus, we have
\(\mathrm{E}(Y) = \kappa^{\prime} = \exp{\phi} = \lambda\) and
\(\mathrm{var}(Y) = \kappa^{\prime \prime}(\phi) = \exp{\phi} = \lambda\).
Consequently, for the Poisson distribution the dispersion index equals
\(1\) \(\forall \lambda\). In the Poisson case is easy to show that
\(\mathrm{ZI} = 0\) and \(\mathrm{HT} \to 0\) when \(y \to \infty\).

Thus, consider a cross-section dataset, \((y_i, x_i)\),
\(i = 1,\ldots, n\), where \(y_i\)'s are i.i.d. realizations of \(Y_i\)
according to a Poisson distribution. The Poisson regression models is
defined by \(Y_i \sim P(\lambda_i)\), with
\(\lambda_i = g^{-1}(\boldsymbol{x_i}^{\top} \boldsymbol{\beta})\). In
this notation, \(\boldsymbol{x_i}\) and \(\boldsymbol{\beta}\) are
(\(q \times 1\)) vectors of known covariates and unknown regression
parameters, respectively. Moreover, \(g\) is a standard link function,
for which here we adopt the logarithm link function, but potentially any
other suitable link function could be adopted. Given the properties of
the Poisson distribution is quite clear that it can deal only with
equidispersed data and has no flexibility to deal with zero-inflation
and/or heavy tail count data. In fact, the presented indexes were
proposed in relation to the Poisson distribution in order to highlight
its limitations.

\section{Gamma-Count distribution}\label{gamma-count-distribution}

The Poisson distribution as presented in \ref{SS} follows directly from
the natural exponential family and thus fits in the generalized linear
models\textasciitilde{}(GLMs) framework. Alternatively, the Poisson
distribution can be derived by assuming independent and exponentially
distributed times between events\textasciitilde{}\citet{Zeviani:2014}.
This derivation allows for a flexible framework to specify more general
count models to deal with under and overdispersed count data.

As point out by \citet{Winkelmann:2003} the distributions of the arrival
times determine the distribution of the number of events.
Following\textasciitilde{}\citet{Winkelmann:2003}, let
\({\tau_k, k \in \mathbb{N}}\) denote a sequence of waiting times
between the \((k-1)\)th and the \(k\)th event. Then, the arrival time of
the \(y\)th event is given by \(\nu_y = \sum_{k = 1}^{y} \tau_k\), for
\(y = 1, 2, \ldots\). Furthermore, denote \(Y_T\) the total number of
events in the open interval between \(0\) and \(T\). For fixed \(T\),
\(Y_T\) is a count variable. Indeed, from the definitions of \(Y_T\) and
\(\nu_y\) we have that \(Y_T < m\) iff \(\nu_y \ge T\), which in turn
implies \(P(Y_T < y) = P(\nu_y \ge T) = 1 - F_y(T)\), where \(F_y(T)\)
denotes the cumulative distribution function of \(\nu_y\). Furthermore,

\begin{eqnarray}
\label{DURATION}
P(Y_T = y) &=& P(Y_T < y+1) - P(Y_T < y)
       &=& F_y(T) - F_{y+1}(T).
\end{eqnarray}

Equation\textasciitilde{}\ref{DURATION} provides the fundamental
relation between the distribution of arrival times and the distribution
of counts. Furthermore, this type of specification allows to derive a
rich class of models for count data by choosing a distribution for the
arrival times. In this material, we shall explore the Gamma-Count
distribution which is obtained by specifying the arrival times
distribution as Gamma distributed.

Let \(\tau_k\) be identically and independently gamma distributed, with
density distribution (dropping the index \(k\)) given by

\begin{equation}
f(\tau; \alpha, \beta) = \frac{\beta^{\alpha}}{\Gamma(\alpha)} \tau^{\alpha-1} \exp{-\beta \tau}, \quad \alpha, \beta \in \mathbb{R}^{+}.
\end{equation}

In this parametrization \(\mathrm{E}(\tau) = \alpha/\beta\) and
\(\mathrm{var}(\tau) = \alpha/\beta^2\). Thus, by applying the
convolution formula for gamma distributions is easy to see that the
distribution of \(\nu_y\) is given by

\begin{equation}
f_y(\nu; \alpha, \beta) = \frac{\beta^{y\alpha}}{\Gamma(y\alpha)} \nu^{y\alpha-1} \exp{-\beta \nu}.
\end{equation}

To derive the new count distribution, we have to evaluate the cumulative
distribution function, which after the change of variable
\(u = \beta \alpha\) can be written as

\begin{equation}
\label{INTEGRAL}
F_y(T) = \frac{1}{\Gamma(y\alpha} \int_0^{\beta T} u^{n\alpha -1} \exp{-u}du, quad y - 1, 2, \ldots,
\end{equation}

where the integral is the incomplete gamma function. We denote the right
side of \ref{INTEGRAL} as \(G(\alpha y, \beta T)\). Thus, the number of
event occurrences during the time interval \((0,T)\) has the
two-parameter distribution function

\begin{equation}
\label{MASSFUNCTION}
P(Y = y) = G(\alpha y, \beta T) - G(\alpha (y + 1), \beta T),
\end{equation}

for \(y = 0, 1, \ldots\), where \(\alpha, \beta \in \mathbb{R}^+\).
\citet{Winkelmann:1995} showed that for integer \(\alpha\) the
probability mass function defined in \ref{MASSFUNCTION} is given by

\begin{equation}
P(Y = y) = \exp{-\beta T} \sum_{i = 0}^{\alpha -1} \frac{(\beta T)^{\alpha y + i}}{\alpha y + i}! , \quad y = 0, 1, 2, \ldots .
\end{equation}

For \(\alpha = 1\), \(f(\tau)\) is the exponential distribution and
\ref{MASSFUNCTION} clearly simplifies to the Poisson distribution.

\bibliography{config/rmcd.bib}


\end{document}
